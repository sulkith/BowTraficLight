\documentclass[12pt,a4paper,final]{article}
\usepackage[utf8]{inputenc}
\usepackage[german]{babel}
\usepackage[T1]{fontenc}
\usepackage{amsmath}
\usepackage{amsfonts}
\usepackage{amssymb}
\usepackage{lmodern}
\usepackage[left=2cm,right=2cm,top=2cm,bottom=2cm]{geometry}
\title{Anleitung für BowTraficLight}
\date{\vspace{-10ex}}
\begin{document}
\pagenumbering{gobble}
\maketitle
\section{Grundlegende Bedienung}
\subsection{Empfänger}
Zum Inbetriebnehmen des Empfänger muss die Powerbank eingesteckt werden. Anschließend startet der Initialisierungsprozess. Nach Abschluss der Initialisierung kann der Empfänger verwendet werden.
\subsection{Sender}
Der Sender ist bereits betriebsbereit. Zum Starten des Schiessprogramms muss der Knopf gedrückt werden.\\
Zum Abbruch muss der Knopf für ca. 20 Sekunden gehalten werden. Es ertönt ein mehrmaliger kurzer Signalton vom Empfänger und der Ablauf wird abgebrochen.
\section{Einstellungsmöglichkeiten}
Der Empfänger besitzt 4 kleine Schalter. Bis auf Schalter 4 müssen sich alle Schalter bereits bei der Initialisierung des Empfängers in Zielstellung befinden. Eine nachträgliche Änderung hat keinen Effekt.
\subsection{Schalter 1 -- 6 Pfeilmodus}
Schalter 1 wählt zwischen den verschiedenen Schiesszeiten. Wenn der Schalter auf On steht sind 4 Minuten ausgewählt. Dies entspricht 6 Pfeilen. Im Zustand Off beträgt die Schiesszeit nur 2 Minuten, das entspricht drei Pfeilen.\\
Beim Starten des Geräts sind zwei Signaltöne zu hören, wenn der Timer auf 4 Minuten steht, bei 2 Minuten ertönt nur ein Signalton.
\subsection{Schalter 2 -- Gruppenmodus}
Mit diesem Schalter in der On Position kann der Gruppenmodus aktiviert werden. Damit wird nach dem ersten Lauf automatisch nach 20 Sekunden Pause ein weiterer Lauf für die zweite Gruppe gestartet.\\
Der Gruppenmodus wird beim Einschalten durch zusätzliche drei Signaltöne signalisiert.
\subsection{Schalter 3 -- Blaues Licht}
Mit diesem Schalter kann die Funktion aktiviert werden, dass das Licht nach Beendigung eines Laufes auf Blau umschaltet, um zu signalisieren, dass das Gerät inaktiv ist.\\
Bitte Beachten: blaues Licht ist nicht FITA regelkonform.
\subsection{Schalter 4 -- Signalton}
Mit Schalter 4 kann der Signalton ein- bzw. ausgeschaltet werden.

\end{document}